% This is a template for the writeup of your final project.  Note:
% sample-paper.tex is a sample paper created using this template.
% It's a reasonable example of a ten-ish page .tex paper.
%
% To Compile this, do the following:
% > pdflatex sample-paper
% > bibtex sample-paper
% > pdflatex sample-paper
% > pdflatex sample-paper

% definitely use the article class and 11 pt font.
\documentclass[11pt]{article}

% definitely use fullpage package
\usepackage{fullpage}
\usepackage{amsmath,amsthm,amssymb,amsfonts,mathtools,textcomp,enumerate}
\DeclareMathOperator*{\argmax}{\arg\!\max}

\newcommand{\Tau}{\mathrm{T}}
\newcommand{\E}{\mathrm{E}}
% Include optional other packages here.

% If you have lots of macros and other new command definitions
% consider creating a separate header.tex file.
% If you use this, uncomment the following line:
%\input{header}

% Next, include a title
\title{Hidden Markov Models}

% Put all author information here.
% Since it's just a final project paper, don't feel like you need
%  to include emails or affiliations.
\author{Colin Crowley \& Julian Segert}

% This suppresses the date.
% if you skip it, the date will be automatically be included
\date{}

% starts the document
\begin{document}

% This sets up the title/author/date information
\maketitle

% The abstract is a 1-2 paragraph synopsis of your project and
%  what you accomplished.
\begin{abstract}
\end{abstract}

% I like to keep each section of a paper in a different .tex file,
%  but you're certainly under no obligation to.
% The \input command essentially cut'n'pastes content from e.g.
%  intro.tex into this .tex file

%\section{Introduction} \label{sec:intro}



\paragraph{Markov Models}
A Markov Model is a class of model where the system being studied undergoes random and independent state changes. Markov Models are said to obey the ``Markov Property,'' that is, the probability of the system changing states is dependent only on the current state of the system independent of previous states. That is to say
    \begin{align*}
      P[q_t &= S_i | q_{t-1} = S_i, q_{t-2} = S_k, ...]\\
      &= P[q_t = S_i | q_{t-1} = S_i]
    \end{align*}
where $q_t$ is the state of the system at time $t$ and $S_i$ is the
$i$th possible state for the system. This memorylessness gives Markov
Models some very nice properties for proabilistic analysis that allows
for efficient modeling of otherwise very computationally intensive
systems. Many important systems of study also follow the Markov Property
by nature, such as the Brownian Motion of atoms or the dynamics of
allele frequencies in a population. The independence of events also
allows us to use many theorems from discrete probability that we have
seen previously in this course.

\paragraph{Hidden Markov Models}
Here we look at one variant of Markov Models, the so-called Hidden Markov Models or HMMs. These HMMs are considered ``Hidden'' because the system's state cannot be observed at any time. Instead, at each timestep, the system emits one of several possible observations (or symbols) $O_i$. Each state has its own probability distribution for which observation to emit \cite{Rabiner89}. (Note: continuous time Hidden Markov Models do exist, but we restrict our analysis to discrete time variants).

The structure of a Hidden Markov Model is composed of two matrices: one that holds the state change transition probabilities, and one that holds the emission probabilities of each of the obseravations for each state. A HMM must also have a probability distribution for selecting a starting state $\pi$ \[
  \pi_i = P[q_1 = S_i], \ \ \ 1 \leq i \leq N.
\] For a concrete example, we'll look at a student's course selection at a hypothetical college. This is a very small school, with only three departments: computer science, biology, and linguistics. Our student starts out as a major in one of these three departments with probability \[
  \pi = \begin{cases}
            0.5 & \text{Computer Science}\\
            0.3 & \text{Biology}\\
            0.2 & \text{Linguistics}
          \end{cases}
\] She will change majors at each semester according to the state transition matrix \[
  A = \{a_{ij}\} = \begin{bmatrix}
                      0.8 & 0.1 & 0.1\\
                      0.2 & 0.7 & 0.1\\
                      0.1 & 0.3 & 0.6
                    \end{bmatrix}
\] Students at this school do not tell their major to anyone. Instead, one has to guess their major by asking their plans after graduation. Every student says they either want to go to medical school, graduate school, law school, or find a job right away. The probabilities of saying any of these options depends on their current major according to the observation matrix \[
  B = P[v_k \text{ at } t|q_t = S_j] = \begin{bmatrix}
                                        0.05 & .4 & 0.05 & 0.5\\
                                        0.8 & 0.1 & 0.05 & 0.05\\
                                        0 & 0.95 & 0.05 & 0
                                      \end{bmatrix}
\] With $1 \leq j \leq N, 1 \leq k \leq M$ for $N =$ number of states and $M =$ number of possible observations. (Note: $B$ is often represented as a function $b_j(k) = P[v_k \text{ at } t|q_t = S_j]$, but we represent it as a matrix and implement it as such).


\tikzset{statenode/.style={circle, draw, very thick, minimum size=20mm}}
\tikzset{observnode/.style={rectangle, draw, very thick, minimum size=20mm}}
%\tikzset{loop/.style={looseness=10}}
\begin{figure}
	\begin{tikzpicture}
		%states
		\node[statenode] (CS) at (2,0) [draw, align=center] {Computer \\ Science};
		\node[statenode] (bio) at (8,0) {Biology};
		\node[statenode] (ling) at (14, 0) {Linguistics};


		%transition probabilities
		 \path[->] (CS) edge  [loop above] node {0.8} ();
		 \path[->] (CS) edge [bend left] node [fill=white] {0.1} (bio);
		 \path[->] (CS) edge [looseness = 1.4] node [fill=white] {0.1} (ling);

		 \path[->] (bio) edge [bend left] node [fill=white] {0.2} (CS);
		 \path[->] (bio) edge [loop above] node {0.7} ();
		 \path[->] (bio) edge [bend left] node [fill=white] {0.1} (ling);


		 \path[->] (ling) edge [bend left, looseness = 1.2] node [fill=white] {0.1} (CS);
		 \path[->] (ling) edge [bend left] node [fill=white] {0.3} (bio);
		 \path[->] (ling) edge [loop above] node {0.6} ();



		 %emission probabilities
		 \node[observnode] (med) at (0, -9) [draw, align=center] {Medical \\ School};
		 \node[observnode] (grad) at (5, -9) [draw, align=center] {Graduate \\ School};
		 \node[observnode] (law) at (10, -9) [draw, align=center] {Law \\ School};
		 \node[observnode] (job) at (15, -9) [draw, align=center] {Job};

		 \path[->, very thick] (CS) edge [bend right] node [fill=white] {0.05} (med);
		 \path[->, very thick] (CS) edge [bend right, looseness = 0.8] node [fill=white] {0.4} (grad);
		 \path[->, very thick] (CS) edge [bend right] node [fill=white, looseness = 0.6] {0.05} (law);
		 \path[->, very thick] (CS) edge [bend right] node [fill=white, looseness = 0.4] {0.5} (job);


		\path[->, very thick] (bio) edge [bend right, looseness = 0.3] node [fill=white] {0.8} (med);
		\path[->, very thick] (bio) edge [bend right, looseness = 0.8] node [fill=white] {0.1} (grad);
		\path[->, very thick] (bio) edge [bend left, looseness = 0.8] node [fill=white] {0.05} (law);
		\path[->, very thick] (bio) edge [bend left, looseness = 0.8] node [fill=white] {0.05} (job);



		\path[->, very thick] (ling) edge [bend left, looseness = 0.5] node [fill=white] {0.95} (grad);
		\path[->, very thick] (ling) edge [bend left, looseness = 0.8] node [fill=white] {0.05} (law);



	\end{tikzpicture}
	Figure 1. A Trellis diagram for the Hidden Markov Model example outlined in the introduction. Circles are possible states and squares are possible observations. Thin line represent state change probabilities and thick lines represent emission probabilities.
\end{figure}


\paragraph{Motivation and Applications}
Hidden Markov Models are extremely useful for modeling complex probabilistic processes in the real world. The above example may not be terribly realistic, but that's mostly because it is simplified for illustration purposes (I'm sure some linguistics majors can find jobs). HMMs become useful when systems grow to be increasingly complex, and analyses that cannot assume independence of events become computationally intractable.

HMMs have many applications in bioinformatics, especially in computational genome annotation. Many algorithms exist to iterate over a newly assembled genome and attempt to determine the state of each segment of DNA as one of the possible genome elements (i.e. exon, intron, non-coding region, ...) \cite{Stanke03}. The true state of each position is unknown, and model parameters can only be estimated. Because of the large size of genomes, this is one example of when the relative computational simplicity of HMMs is beneficial.

Moreover, HMMs are used heavily in computational linguistics both for
audio speech recognition and part-of-speech recognition. Audio signals
can be discretized by taking short segments (i.e. 10 milliseconds) and
looking at the most significant coefficients from the Fourier transform.
In this way, a HMM can be trained to identify the discrete phonemes of a
spoken language with a high degree of accuracy and little training data
\cite{Gales07}. Higher-order HMMs can then identify which part of speech
a given word is most likely to be given what part of speech the previous
word is \cite{Hull92}.

Once processes are modeled as HMMs, one can clearly see the motivation
behind calculating certain properties, such as the most likely sequence
of states, or how to find unknown model parameters. In the following
section, we will provide algorithms for determining these properties.

%Don't forget to fill in dummy citations!!

\section{Analysis of Algorithms}
\subsection{Notation}

We will begin by defining the notation used to analyze Hidden Markov
Models. The set of states within the model is denoted as

\begin{equation}
  S = \{S_1, S_2, \ldots, S_N \}
\end{equation}

where $N$ is the number of states. The symbols of the model are likewise
denoted as

\begin{equation}
  V = \{V_1, V_2, \ldots, V_M \}
\end{equation}

where $M$ is the number of possible symbols. The current time is denoted $t$
which ranges from $t=1, \ldots, \Tau$, and the state at the current
time is denoted $q_t$. We need a separate letter for the state at the
current time because the letter $S$ is indexed as a list of different
states and we want $q$ to mean the \emph{sequence of states over time}.
Next we will define the two matrices that form the core of the model.
The first matrix defines that probability that the next state will be
$S_j$ given that the current state is $S_i$, and is denoted

\begin{equation}
  A = \{a_{ij}\}, \quad a_{ij} = \Pr[q_{t+1} = S_j \vert q_{t} = S_i],
  \quad 1 \leq i,j \leq N
\end{equation}

The next matrix is the sets the probability that a particular symbol
will be observed given that the model is in the state $S_i$, and is
written

\begin{equation}
  B = \{b_j(k)\}, \quad b_j(k) = \Pr[v_k\ \textrm{is observed at time}\ t
  \vert q_t = S_j], \quad 1 \leq j \leq N, \quad 1 \leq k \leq M
\end{equation}

While it computationally more convenient to conceptualize $B$ as a
matrix, we can also think of it as a probability distribution of $V$
over $S$. That is the reason why $b_j(k)$ is written as a function instead of
with subscripts like $a_{ij}$. Hidden Markov Models, instead of having a
starting state, have an initial probability distribution over $S$. This
distribution is denoted $\pi$.

\begin{equation}
  \pi_i = \Pr[q_1 = S_i], \quad 1 \leq i \leq N
\end{equation}

Finally, we have a list of observations denoted

\begin{equation}
  O = O_1\ O_2\ \ldots\ \ ,O_\Tau
\end{equation}

We will often begin with a list of observations $O$ and the goal will be
to compute other parameters of the model. For example, in the Viterbi
algorithm we are given $O$ and compute the most likely sequence of
states $q_1,\ \ldots,\ q_\Tau$. Therefore, it is helpful to
conceptualize $O$ as the ``input'' and the underlying states or the
optimal model as the ``goal.''

\subsection{Forwards-Backwards}

The goal of the forwards-backwards algorithm, and the first obvious goal
of a Hidden Markov Model, is finding the likelihood that a particular sequence of
observations will happen. When thinking procedurally it is helpful to think of $O$ as our input,
the rest of the model as the information we can access, and the
probability of $O$ as the output. In notational terms, we are looking for $\Pr[O
\vert \lambda]$ where $\lambda$ are the parameters of the model $\lambda
= (A, B, \pi)$. (We will often need to find probabilities that are
conditional on $A$, $B$, and $\pi$, so we will use $\lambda$ for ease of
notation.) The forwards-backwards algorithm works by recursively finding
the probabilities of a partial series of observations. Let the ``forward
probability'' be the following.

\begin{equation}
\alpha_t(i) = \Pr[O_1\ O_2\ \ldots\ O_t\ \textrm{is observed \emph{and}}\ q_t = S_i \vert \lambda]
\end{equation}

The letter $\alpha$ is the probability that the first $t$ elements
of $O$ are observed and the state at the end of that partial observation
is $S_i$. The probability $\Pr[O \vert \lambda]$ which we are
looking for can be expressed as the following sum.

\begin{equation}
  \Pr[O \vert \lambda] = \sum_{i=1}^N \alpha_\Tau(i)
\end{equation}

Remember that $\Tau$ is the time of the last observation. Let us prove
this by using the definition of $\alpha$ and the Markov property, which
tells us that the next state.

\begin{proof}
  We have the sum
  \begin{align*}
    \sum_{i=1}^N \alpha_t(i) &= \sum_{i=1}^N \Pr[O\ \cap\ q_t = S_i \vert \lambda]\\
    \intertext{where $O = O_1\ O_2\ \ldots\ O_\Tau$. Rewriting in terms of conditional probability gives us}
    &= \sum_{i=1}^N \Pr[O\ \vert \lambda] \Pr[q_\Tau = S_i \vert O,\lambda]
    \intertext{Because the first probability within sigma does not vary
    with $i$, we can pull it outside to get}
    &=  \Pr[O\ \vert \lambda] \sum_{i=1}^N \Pr[q_\Tau = S_i \vert O,\lambda]
    \intertext{Because the model cannot be in more than one state at
    any one time, the events that $q_\Tau = S_i$ given that $O,\lambda$
    happened for $1 \leq i \leq N$ are independent. And so the sum of
    their probabilities is equal to the probability of their union.
    Furthermore, because the model must be in some state at any one
    time, the probability of their union is equal to 1. Thus we have}
    &= \Pr[O\ \vert \lambda]
  \end{align*}
\end{proof}

We can now use alpha to output the probability $\Pr[O\ \vert \lambda]$,
but the reason that this approach is useful is because $\alpha$ has a recursive
definition that leads to an efficient calculation of that probability.
Specifically, we can define alpha as follows.

\begin{equation}
  \alpha_1(i) = \pi_i b_i(O_1)\\
\end{equation}

\begin{proof}
  \begin{align*}
    \alpha_1(i) &= \Pr[O_1\ \cap\ q_1=S_i]\\
    \intertext{These events are independent, so}
    &= \Pr[O_1] \Pr[q_1=S_i]\\
    &= \pi_1 b_i(O_1)\\
  \end{align*}
\end{proof}

Let $t=1$ be the base case. For the recursive step, we can define
$\alpha_t(i)$ as

\begin{equation}
  \alpha_t(j) = \left( \sum_{i=1}^N \alpha_{t-1}(i) a_{ij} \right)
  b_t(O_j)
\end{equation}

\begin{proof}
  \begin{align*}
    \alpha_t(j) &= \Pr[O_t\ \cap\ q_t=S_j]\\
                &= \Pr[O_1\ O_2\ \ldots\ O_{t-1}\ \cap q_t=S_j \cap O_t]\\
                &= \Pr[O_1\ O_2\ \ldots\ O_{t-1} \cap q_t=S_j]  \Pr[O_t \vert q_t=S_j,\ O_1\ O_2\ \ldots\ O_{t-1}]\\
                &= \Pr[O_1\ O_2\ \ldots\ O_{t-1} \cap q_t=S_j] b_j(O_t)\\
                &= \Pr[O_1\ O_2\ \ldots\ O_{t-1} \cap q_t=S_j] b_j(O_t)\\
    \intertext{The probability in the equation directly above can be
    expressed as the probability that the first $t-1$ observations were
    made which is $\sum_{i=1}^N \alpha_{t-1}(i)$ times the probability
    that whatever state $\alpha_{t-1}$ leaves us in will transition to
    $S_i$ which is $a_{ij}$. Since these two are independent, they can
    both be expressed as a product in the following equation.}
                &= \left( \sum_{i=1}^N \alpha_{t-1}(i) a_{ij} \right) b_t(O_j)
  \end{align*}
\end{proof}

This recursive definition allows us to make a two dimensional array of
values for alpha across $t$ and $i$, and fill it in with one pass using
dynamic programming. The algorithm works as follows.
\begin{enumerate}[1)]
    \item Initialize a two dimensional array with columns that represent
      $i$ (the possible states) and rows that represent $t$ (the time).
    \item Loop through the first row and initialize it with $\pi_i b_i(O_1)$
    \item Have an outer loop through the rows (besides the first one)
      and an inner loop through the columns. Set each entry to
      $\alpha_t(j) = \left( \sum_{i=1}^N \alpha_{t-1}(i) a_{ij} \right)
      b_t(O_j)$
\end{enumerate}

This algorithm requires $N^2\Tau$ steps: $N\Tau$ steps for each entry in the
array, and while filling each entry, there are $N$ steps to compute the
sum that it involves. This $O(N^2\Tau)$ runtime algorithm is quite
efficient when compared with the naive approach of iterating over every possible sequence.

Let us now define the ``Backwards Probability.''

\begin{equation}
  \beta_t(i) = \Pr[O_{t+1}\ O_{t+2}\ \ldots\ O_\Tau \vert q_t = S_i,
  \lambda]
\end{equation}

That the $q_t = S_i$ event is a condition in the backwards probability
will make calculations later on easier. We can define $\beta$ in a
similar way to $\alpha$, and the probability returned by forwards
backwards can also be found with $\beta$, but because we already have a way
to do it with $\alpha$, we won't prove the properties of $\beta$ in as
much detail. $\beta$ will be used as a tool to derive other probabilities.
It can be calculated using a similar recursive definition to $\alpha$.

\begin{equation}
  \beta_\Tau(i) = 1\\
\end{equation}
\begin{equation}
  \beta_t(i) = \sum_{j=1}^N a_{ij} b_j(O_t+1) \beta_{t+1}(j)
\end{equation}

One can follow a slightly modified version of the above algorithm for
finding $\alpha$ and get a $O(N^2\Tau)$ runtime algorithm for finding
$\beta$ as well.

\subsection{Viterbi Algorithm}

The aim of this algorithm is to find the sequence of states that
maximizes the likelihood of the inputed sequence of observations being
observed. Let's define the probability $\delta$ as follows.

\begin{equation}
  \delta_1(i) = \pi_i b_i(O_1)\\
\end{equation}

\begin{equation}
  \delta_t(j) = \max_i[\delta_{t-1}(i)\ a_{ij}\ ] b_j(O_t)
\end{equation}

Note that $\delta$ is identical to $\alpha$ except in the recursive
definition, instead of summing over the previous calls, $\delta$ maximizes
over them. While summing gave the total probability that we observe the
partial sequence of events and end up in state $S_i$ at time $t$,
maximizing gives the maximum probability for a single sequence of
states that we get the sequence of observations and end up in state
$S_i$ at time $t$. In formal notation

\begin{equation}
  \delta_t(i) = \max_{q_1,\ q_2,\ \ldots\ q_{t-1}} \Pr[O_1\ O_2\ \ldots\
  O_t\ \cap q_t = S_i \cap q_1,\ q_2,\ \ldots\ q_{t-1}]
\end{equation}

While summing up $\alpha_\Tau$ over $i$ gave the total probability of the
observation, maximizing $\delta_\Tau$ over $i$ gives the probability of
the observation with the optimal state sequence. (CITATION)

\begin{equation}
  \max_i[\delta_\Tau(i)]
\end{equation}

We can now follow the same dynamic programming algorithm for $\alpha$ to
find the probability that comes from the optimal state sequence,
however, we will also need to keep track of the state sequence at each
point to be able to recover it in the end. The following algorithm does
this with two two dimensional arrays.

\begin{enumerate}
    \item Make two 2-Dimensional arrays with columns that
      represent $i$ (the state) and rows that represent $t$ (the time).
      Call one \emph{delta} and call one \emph{psi}.
    \item Initialize the first row of delta with $\pi_i b_i(O_1)$.
    \item Have an outer loop through the rows (besides the first one)
      and an inner loop through the columns. Set \emph{delta}$[t][i] =
      \max_k[\delta_{t-1}(k)\ a_{ki}\ ] b_i(O_t)$. Set \emph{psi}$[t][i]
      = \argmax_k[\delta_{t-1}(k)\ a_{ki}\ ]$
    \item Now pick the best path ending by selecting \emph{path}$[\Tau] =
      \argmax_k[\textrm{\emph{delta}}[\Tau][k]]$. And follow that path
      backwards to create \emph{path}$[t] =
      $\emph{psi}$[$\emph{path}$[t+1]]$
\end{enumerate}

ADD SOME MORE HERE-----------------------------------------

\subsection{Baum-Welch Algorithm}

This algorithm finds the $\alpha$, $\beta$, and $\pi$ that predict a
given sequence of events with the locally maximal probability. There is
no know way to find the globally maximal probability. (CITATION) This is
essentially how one can train a hidden Markov Model with observed data.

The Baum-Welch Algorithm uses two probability functions other than
$\alpha$ and $\beta$ that we will define first. The function
$\gamma_t(i)$ is the probability that the model is in state $S_i$ at
time $t$ and is written formally as

\begin{equation}
  \gamma_t(i) = \Pr[q_t = S_i \vert O, \lambda]
\end{equation}

The function $\gamma_t(i)$ can be calculated by finding the probability
of the observation and that $q_t = S_i$ and dividing it by the
probability that the observation happens. We can there for express it as

\begin{equation}
  \gamma_t(i) = \frac{\alpha_t(i)\beta_t(i)}{\sum_{k=1}^N
  \alpha_\Tau(k)}
\end{equation}

It is easy to reason that this is the correct expression because
\begin{align*}
  \alpha_t(i) \beta_t(i) &= \Pr[O_1\ O_2\ \ldots\ O_t \cap q_t = S_i \vert
  \lambda] \Pr[O_1\ O_2\ \ldots\ O_t \vert q_t = S_i, \lambda]\\
    &= \Pr[O_1\ O_2\ \ldots\ O_\Tau \cap q_t = S_i \vert \lambda]
  \intertext{and}
  \sum_{k=1}^N \alpha_\Tau(k) &= \Pr[O \vert \lambda]
\end{align*}

The second probability function that we will need is $\xi_t(i,j)$,
which is the probability that $q_t = S_i$ and $q_{t+1} = S_j$ given the
observations and the model. This probability of a transition will,
naturally, be helpful when computing the optimal transition matrix.

\begin{equation}
  \xi_t(i,j) = \Pr[q_t = S_i \cap q_{t+1} = S_j \vert O,\lambda]
\end{equation}

We can use the same logic as we did for $\gamma$, only this time also
factoring in the probability that $q_{t+1} = S_j$ with the following
equation.

\begin{equation}
  \xi_t(i,j) = \frac{\alpha_t(i)\ a_{ij}\ b_j(O_{t+1})\
  \beta_{t+1}(j)}{\sum_{k=1}^N \alpha_\Tau(k)}
\end{equation}

The Baum-Welch algorithm uses expected values of transitions and
emissions found from the current parameters to find better parameters.
Specifically the following. (CITATION)

\begin{equation}
  \pi_i = \Pr[q_1 = S_i \vert O,\lambda] = \gamma_1(i)
\end{equation}

\begin{equation}
  a_{ij} = \frac{\E[\textrm{number of transitions from $S_i$ to
  $S_t$}]}{\E[\textrm{number of transitions from $S_i$}]} =
  \frac{\sum_{t=1}^{N-1} \xi_t(i,j)}{\sum_{t=1}^{N-1} \gamma_t(i)}
\end{equation}

\begin{equation}
  b_i(k) = \frac{\E[\textrm{number of times in state $i$ and emitting symbol
  $k$}]}{\E[\textrm{number of times in state $i$}]} = \frac{\sum_{t\ \textrm{s.t.} O_t = v_k} \gamma_t(i)}{\sum_{t=1}^N \gamma_t(i)}
\end{equation}

Each one of these calculation finds the values of the parameters
\emph{conditioned on the observation and the current parameters}. This
is clear to see with $\pi$ as it is in the equation, but it is also true
of $A$ and $B$ because, remember, both $\gamma$ and $\xi$ are
conditional on $O$ and $\lambda$. With these new parameters, which
improve the likelihood of observing $O$, we can repeat this process.

%\input{prelim}
%\input{firstresult}
%\input{secondresult}
%

\section{Conclusion}
We have seen that Hidden Markov Models are useful tools for modeling
diverse probabilistic systems. We have presented and analyzed algorithms
for computing useful properties of HMMs with relatively fast runtime.
Included is a Python implementation of a HMM and the algorithms
discussed above, along with relevant test code. While useful, HMMs do
have their limitations. For example, in many situations, one cannot
assume the Markov Property or doing so would limit the predictive
power of the model. HMMs are most useful in situations like speech
recognition or genetics where we have some a priori knowledge of the
internal parameters.



%--------------------------------- bibliography -------------------------
% Thus is a common way to do bibliographies called bibtex
%  write bibtex entries in a .bib file (ours was super.bib)
% Let me know if you have questions on how to this, and/or
%  try to mimic what we do here
\bibliographystyle{plain}
\bibliography{super}

% Here's how to do appendices
\appendix
%\input{maintechnicalstuff}
%\input{otherstuff}

\end{document}
